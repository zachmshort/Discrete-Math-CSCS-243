\documentclass{article}

\usepackage[english]{babel}

\usepackage[letterpaper,top=2cm,bottom=2cm,left=3cm,right=3cm,marginparwidth=1.75cm]{geometry}

\usepackage{amsmath}
\usepackage{graphicx}
\usepackage[colorlinks=true, allcolors=blue]{hyperref}

\renewcommand{\thesubsection}{\alph{subsection}.}

\title{CSCI-243 HW-1}
\author{Zachary Short}

\begin{document}
\maketitle

\section{Express the following statements in propositional logic:}
    \subsection{It is snowing and you wear mittens.}
    \subsection{If it is not snowing, you do not go sledding.} 
    \subsection{If it is snowing, you go sledding or wear mittens, but not both.}
    \subsection{You go sledding whenever it is not snowing.}
\section{Assume the proposition "If I am a senior at W&M, then I have declared a major" is true.}
    \subsection{State the converse of this proposition.} 
    \subsection{State the inverse of this proposition.} 
    \subsection{State the contrapositive of this proposition.} 
    \subsection{For each of the above, state whether it is always true, sometimes true, or never true.} 
\section{Construct a truth table for each of the following compound propositions:}
    \subsection{} 
    \subsection{} 
\section{Answer the following questions concerning logic circuits:}
    \subsection{What compound proposition does the following circuit represent (do not simplify)?} 
    \subsection{Is the above satisfiable? Prove your answer with a truth table(using the unsimplified compound proposition).} 
\end{document}